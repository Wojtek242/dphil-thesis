%*******************************************************************************
%*********************************** First Chapter *****************************
%*******************************************************************************

\chapter{Introduction}  %Title of the First Chapter

\ifpdf
    \graphicspath{{Chapter1/Figs/Raster/}{Chapter1/Figs/PDF/}{Chapter1/Figs/}}
\else
    \graphicspath{{Chapter1/Figs/Vector/}{Chapter1/Figs/}}
\fi


The field of ultracold gases has been a rapidly growing field ever
since the first Bose-Einstein condensate was obtained in 1995. This
new quantum state of matter is characterised by a macroscopic
occupancy of the single particle ground state at which point the whole
system behaves like a single quantum object. This was revolutionary as
it enabled the study of coherent properties of macroscopic systems
rather than single atoms or photons. Furthermore, the advanced state
of laser cooling and manipulation technologies meant that the degree
of control and isolation from the environment was far greater than was
possible in condensed matter systems. Initially, the main focus of the
research was on the properties of coherent matter waves, such as
interference properties, long range phase coherence, or quantised
vortices. Fermi degeneracy in ultracold gases was obtained shortly
afterwards opening a similar field for fermions.

In 1998 it was shown that a degenerate ultracold gas trapped in an
optical lattice is a near-perfect realisation of the Bose-Hubbard
model and in 2002 it was already demonstrated in a ground-breaking
experiment. The Bose-Hubbard Hamiltonian was already known in the
field of condensed matter where it was considered a simple toy
model. Despite its simplicity the model exhibits a variety of
different interesting phenomena such as the quantum phase transition
from a delocalised superfluid state to a Mott insulator as the on-site
interaction is increased above a critical point. In contrast to a
thermodynamic phase transition, a quantum phase transition is driven
by quantum fluctuations and can occur at zero temperature. The ability 




The modern field of ultracold gases is successful due to its
interdisciplinarity [1, 2]. Originally condensed matter effects are
now mimicked in controlled atomic systems finding applications in
areas such as quantum information processing (QIP). A really new
challenge is to identify novel phenomena which were unreasonable to
consider in condensed matter, but will become feasible in new systems.
One such direction is merging quantum optics and many-body physics [3,
  4]. The former describes delicate effects such as quantum
measurement and state engineering, but for systems without strong
many-body correlations (e.g. atomic ensembles).  In the latter,
decoherence destroys these effects in conventional condensed
matter. Due to recent experimental progress, e.g. Bose-Einstein
condensates (BEC) in cavities [5–7], quantum optics of quantum gases
can close this gap.
