%*******************************************************************************
%*********************************** Third Chapter *****************************
%*******************************************************************************

\chapter{Probing Correlations by Global 
Nondestructive Addressing} %Title of the Third Chapter

\ifpdf
    \graphicspath{{Chapter3/Figs/Raster/}{Chapter3/Figs/PDF/}{Chapter3/Figs/}}
\else
    \graphicspath{{Chapter3/Figs/Vector/}{Chapter3/Figs/}}
\fi


%********************************** %First Section  **************************************

\section{Introduction}

Having developed the basic theoretical framework within which we can
treat the fully quantum regime of light-matter interactions we now
consider possible applications. There are three prominent directions
in which we can apply our model: nondestructive probing, quantum
measurement backaction and quantum optical lattices. Here, we deal
with the first of the three options.

\mynote{adjust for the fact that the derivation of B operator has been moved}
\mynote{update some outdated mentions to previous experiments}
In this chapter we develop a method to measure properties of ultracold
gases in optical lattices by light scattering. We show that such
measurements can reveal not only density correlations, but also
matter-field interference.  Recent quantum non-demolition (QND)
schemes \cite{rogers2014, mekhov2007prl, eckert2008} probe density
fluctuations and thus inevitably destroy information about phase,
i.e.~the conjugate variable, and as a consequence destroy matter-field
coherence. In contrast, we focus on probing the atom interference
between lattice sites. Our scheme is nondestructive in contrast to
totally destructive methods such as time-of-flight measurements. It
enables in-situ probing of the matter-field coherence at its shortest
possible distance in an optical lattice, i.e. the lattice period,
which defines key processes such as tunnelling, currents, phase
gradients, etc. This is achieved by concentrating light between the
sites. By contrast, standard destructive time-of-flight measurements
deal with far-field interference and a relatively near-field one was
used in Ref. \cite{miyake2011}. Such a counter-intuitive configuration
may affect works on quantum gases trapped in quantum potentials
\cite{mekhov2012, mekhov2008, larson2008, chen2009, habibian2013,
  ivanov2014, caballero2015} and quantum measurement-induced
preparation of many-body atomic states \cite{mazzucchi2016,
  mekhov2009prl, pedersen2014, elliott2015}. Within the mean-field
treatment, this enables measurements of the order parameter,
matter-field quadratures and squeezing. This can have an impact on
atom-wave metrology and information processing in areas where quantum
optics already made progress, e.g., quantum imaging with pixellized
sources of non-classical light \cite{golubev2010, kolobov1999}, as an
optical lattice is a natural source of multimode nonclassical matter
waves.

Furthermore, the scattering angular distribution is nontrivial, even
when classical diffraction is forbidden and we derive generalized
Bragg conditions for this situation. The method works beyond
mean-field, which we demonstrate by distinguishing all three phases in
the Mott insulator - superfluid - Bose glass phase transition in a 1D
disordered optical lattice. We underline that transitions in 1D are
much more visible when changing an atomic density rather than for
fixed-density scattering. It was only recently that an experiment
distinguished a Mott insulator from a Bose glass \cite{derrico2014}.

\section{Global Nondestructive Measurement}

As we have seen in section \ref{sec:a} under certain approximations
the scattered light mode, $\a_1$, is linked to the quantum state of
matter via 
\begin{equation}
  \label{eq:a-3}
  \a_1 = C \hat{F} = C \left(\hat{D} + \hat{B} \right),
\end{equation}
where the atomic operators $\hat{D}$ and $\hat{B}$, given by
Eq. \eqref{eq:D} and Eq. \eqref{eq:B}, are responsible for the
coupling to on-site density and inter-site interference
respectively. It crucial to note that light couples to the bosons via
an operator as this makes it sensitive to the quantum state of the
matter.

Here, we will use this fact that the light is sensitive to the atomic
quantum state due to the coupling of the optical and matter fields via
operators in order to develop a method to probe the properties of an
ultracold gas. Therefore, we neglect the measurement back-action and
we will only consider expectation values of light observables. Since
the scheme is nondestructive (in some cases, it even satisfies the
stricter requirements for a QND measurement \cite{mekhov2007pra,
  mekhov2012}) and the measurement only weakly perturbs the system,
many consecutive measurements can be carried out with the same atoms
without preparing a new sample. Again, we will show how the extreme
flexibility of the the measurement operator $\hat{F}$ allows us to
probe a variety of different atomic properties in-situ ranging from
density correlations to matter-field interference.

\subsection{On-site density measurements}

Typically, the dominant term in $\hat{F}$ is the density term
$\hat{D}$, rather than inter-site matter-field interference $\hat{B}$
\cite{mekhov2007pra, rist2010, lakomy2009, ruostekoski2009,
  LP2009}. However, before we move onto probing the interference
terms, $\hat{B}$, we will first discuss typical light scattering. We
start with a simpler case when scattering is faster than tunneling and
$\hat{F} = \hat{D}$. This corresponds to a QND scheme
\cite{mekhov2007prl, mekhov2007pra, eckert2008, rogers2014}. The
density-related measurement destroys some matter-phase coherence in
the conjugate variable \cite{mekhov2009pra, LP2010, LP2011}
$\bd_i b_{i+1}$, but this term is neglected. For this purpose we will
define an auxiliary quantity,
\begin{equation}
  \label{eq:R}
  R = \langle \ad_1 \a_1 \rangle - | \langle \a_1 \rangle |^2,
\end{equation}
which we will call the ``quantum addition'' to light scattering. $R$
is simply the full light intensity minus the classical field intensity
and thus it faithfully represents the new contribution from the
quantum light-matter interaction to the diffraction pattern.

\begin{figure}[htbp!]
  \centering
  \includegraphics[width=\linewidth]{Ep1}
  \caption[Light Scattering Angular Distribution]{Light intensity
    scattered into a standing wave mode from a superfluid in a 3D
    lattice (units of $R/N_K$). Arrows denote incoming travelling wave
    probes. The Bragg condition, $\Delta \b{k} = \b{G}$, is not
    fulfilled, so there is no classical diffraction, but intensity
    still shows multiple peaks, whose heights are tunable by simple
    phase shifts of the optical beams: (a) $\varphi_1=0$; (b)
    $\varphi_1=\pi/2$. Interestingly, there is also a significant
    uniform background level of scattering which does not occur in its
    classical counterpart. }
  \label{fig:Scattering}
\end{figure}

In a deep lattice,
\begin{equation} 
  \hat{D}=\sum_i^K u_1^*({\bf r}_i) u_0({\bf r}_i) \hat{n}_i,
\end{equation} 
which for travelling
[$u_l(\b{r})=\exp(i \b{k}_l \cdot \b{r}+i\varphi_l)$] or standing
[$u_l(\b{r})=\cos(\b{k}_l \cdot \b{r}+\varphi_l)$] waves is just a
density Fourier transform at one or several wave vectors
$\pm(\b{k}_1 \pm \b{k}_0)$. The quadrature, as defined in section
\ref{sec:a}, for two travelling waves is reduced to
\begin{equation} 
  \hat{X}^F_\beta = \sum_i^K \hat{n}_i\cos[(\b{k}_1 - \b{k}_2) \cdot
  \b{r}_i - \beta].
\end{equation} 
Note that different light quadratures are differently coupled to the
atom distribution, hence varying local oscillator phase and detection
angle, one scans the coupling from maximal to zero. An identical
expression exists for $\hat{D}$ for a standing wave, where $\beta$ is
replaced by $\varphi_l$, and scanning is achieved by varying the
position of the wave with respect to atoms. Thus, variance
$(\Delta X^F_\beta)^2$ and quantum addition $R$, have a non-trivial
angular dependence, showing more peaks than classical diffraction and
the peaks can be tuned by the light-atom coupling.

Fig. \ref{fig:Scattering} shows the angular dependence of $R$ for
standing and travelling waves scattering from bosons in a 3D optical
lattice. The isotropic background gives the density fluctuations
[$R = K( \langle \hat{n}^2 \rangle - \langle \hat{n} \rangle^2 )/2$ in
mean-field with inter-site correlations neglected]. The radius of the
sphere changes from zero, when it is a Mott insulator with suppressed
fluctuations, to half the atom number at $K$ sites, $N_K/2$, in the
deep superfluid. There exist peaks at angles different than the
classical Bragg ones and thus, can be observed without being masked by
classical diffraction. Interestingly, even if 3D diffraction
\cite{miyake2011} is forbidden as seen in Fig. \ref{fig:Scattering},
the peaks are still present. As $(\Delta X^F_\beta)^2$ and $R$ are
quadratic variables, the generalized Bragg conditions for the peaks
are $2 \Delta \b{k} = \b{G}$ for quadratures of travelling waves,
where $\Delta \b{k} = \b{k}_0 - \b{k}_1$ and $\b{G}$ is the reciprocal
lattice vector, and $2 \b{k}_1 = \b{G}$ for standing wave $\a_1$ and
travelling $\a_0$, which is clearly different from the classical Bragg
condition $\Delta \b{k} = \b{G}$. The peak height is tunable by the
local oscillator phase or standing wave shift as seen in Fig.
\ref{fig:Scattering}b.

In section \ref{sec:Efield} we have estimated the mean photon
scattering rates integrated over the solid angle for the only two
experiments so far on light diffraction from truly ultracold bosons
where the measurement object was light
\begin{equation} 
  n_{\Phi}= \left(\frac{\Omega_0}{\Delta_a}\right)^2 \frac{\Gamma K}{8}
  (\langle\hat{n}^2\rangle-\langle\hat{n}\rangle^2).
\end{equation} 
The background signal should reach $n_\Phi \approx 10^6$ s$^{-1}$ in
Ref. \cite{weitenberg2011} (150 atoms in 2D), and
$n_\Phi \approx 10^{11}$ s$^{-1}$ in Ref. \cite{miyake2011} ($10^5$
atoms in 3D). These numbers show that the diffraction patterns we have
seen due to the ``quantum addition'' should be visible using currently
available technology, especially since the most prominent features,
such as Bragg diffraction peaks, do not coincide at all with the
classical diffraction pattern.

\subsection{Matter-field interference measurements}

We now focus on enhancing the interference term $\hat{B}$ in the
operator $\hat{F}$. 

Firstly, we will use this result to show how one can probe
$\langle \hat{B} \rangle$ which in MF gives information about the
matter-field amplitude, $\Phi = \langle b \rangle$. 

Hence, by measuring the light quadrature we probe the kinetic energy
and, in MF, the matter-field amplitude (order parameter) $\Phi$:
$\langle \hat{X}^F_{\beta=0} \rangle = | \Phi |^2
\mathcal{F}[W_1](2\pi/d) (K-1)$.

Secondly, we show that it is also possible to access the fluctuations
of matter-field quadratures $\hat{X}^b_\alpha = (b e^{-i\alpha} + \bd
e^{i\alpha})/2$, which in MF can be probed by measuring the variance
of $\hat{B}$. Across the phase transition, the matter field changes
its state from Fock (in MI) to coherent (deep SF) through an
amplitude-squeezed state as shown in Fig. \ref{Quads}(a,b). 

Assuming $\Phi$ is real in MF:
\begin{equation}
  \label{intensity} 
  \langle \ad_1 \a_1 \rangle = 2 |C|^2(K-1)\mathcal{F}^2[W_1](\frac{\pi}{d})
  \times [ ( \langle b^2 \rangle - \Phi^2 )^2 + ( n - \Phi^2 ) ( 1 +n - \Phi^2 ) ]
\end{equation} 
and it is shown as a function of $U/(zJ^\text{cl})$ in
Fig. \ref{Quads}. Thus, since measurement in the diffraction maximum
yields $\Phi^2$ we can deduce $\langle b^2 \rangle - \Phi^2$ from the
intensity. This quantity is of great interest as it gives us access to
the quadrature variances of the matter-field
\begin{equation} 
  (\Delta X^b_{0,\pi/2})^2 = 1/4 + [(n - \Phi^2) \pm
  (\langle b^2 \rangle - \Phi^2)]/2,
\end{equation} 
where $n=\langle\hat{n}\rangle$ is the mean on-site atomic density.

\begin{figure}[htbp!]
  \centering
  \includegraphics[width=\linewidth]{Quads}
  \captionsetup{justification=centerlast,font=small}
  \caption[Mean-Field Matter Quadratures]{Photon number scattered in a
    diffraction minimum, given by Eq. (\ref{intensity}), where
    $\tilde{C} = 2 |C|^2 (K-1) \mathcal{F}^2 [W_1](\pi/d)$.  More
    light is scattered from a MI than a SF due to the large
    uncertainty in phase in the insulator. (a) The variances of
    quadratures $\Delta X^b_0$ (solid) and $\Delta X^b_{\pi/2}$
    (dashed) of the matter field across the phase transition. Level
    1/4 is the minimal (Heisenberg) uncertainty. There are three
    important points along the phase transition: the coherent state
    (SF) at A, the amplitude-squeezed state at B, and the Fock state
    (MI) at C. (b) The uncertainties plotted in phase space.}
	\label{Quads}
\end{figure}

Probing $\hat{B}^2$ gives us access to kinetic energy fluctuations
with 4-point correlations ($\bd_i b_j$ combined in pairs). Measuring
the photon number variance, which is standard in quantum optics, will
lead up to 8-point correlations similar to 4-point density
correlations \cite{mekhov2007pra}. These are of significant interest,
because it has been shown that there are quantum entangled states that
manifest themselves only in high-order correlations
\cite{kaszlikowski2008}.

Surprisingly, inter-site terms scatter more light from a Mott
insulator than a superfluid Eq. \eqref{intensity}, as shown in
Fig. \eqref{Quads}, although the mean inter-site density
$\langle \hat{n}(\b{r})\rangle $ is tiny in a MI. This reflects a
fundamental effect of the boson interference in Fock states. It indeed
happens between two sites, but as the phase is uncertain, it results
in the large variance of $\hat{n}(\b{r})$ captured by light as shown
in Eq. \eqref{intensity}. The interference between two macroscopic
BECs has been observed and studied theoretically
\cite{horak1999}. When two BECs in Fock states interfere a phase
difference is established between them and an interference pattern is
observed which disappears when the results are averaged over a large
number of experimental realizations. This reflects the large
shot-to-shot phase fluctuations corresponding to a large inter-site
variance of $\hat{n}(\b{r})$. By contrast, our method enables the
observation of such phase uncertainty in a Fock state directly between
lattice sites on the microscopic scale in-situ.

\subsection{Mapping the quantum phase diagram}

\begin{figure}[htbp!]  
  \centering
  \includegraphics[width=\linewidth]{oph11}
  \caption[Mapping the Bose-Hubbard Phase Diagram]{(a) The angular
    dependence of scattered light $R$ for a superfluid (thin black,
    left scale, $U/2J^\text{cl} = 0$) and Mott insulator (thick green,
    right scale, $U/2J^\text{cl} =10$). The two phases differ in both
    their value of $R_\text{max}$ as well as $W_R$ showing that
    density correlations in the two phases differ in magnitude as well
    as extent. Light scattering maximum $R_\text{max}$ is shown in (b,
    d) and the width $W_R$ in (c, e).  It is very clear that varying
    chemical potential $\mu$ or density $\langle n\rangle$ sharply
    identifies the superfluid-Mott insulator transition in both
    quantities. (b) and (c) are cross-sections of the phase diagrams
    (d) and (e) at $U/2J^\text{cl}=2$ (thick blue), 3 (thin purple),
    and 4 (dashed blue). Insets show density dependencies for the
    $U/(2 J^\text{cl}) = 3$ line. $K=M=N=25$.}
	\label{fig:SFMI}
\end{figure}

We have shown how in mean-field, we can track the order parameter,
$\Phi$, by probing the matter-field interference using the coupling of
light to the $\hat{B}$ operator. In this case, it is very easy to
follow the superfluid to Mott insulator quantum phase transition since
we have direct access to the order parameter which goes to zero in the
insulating phase. In fact, if we're only interested in the critical
point, we only need access to any quantity that yields information
about density fluctuations which also go to zero in the MI phase and
this can be obtained by measuring
$\langle \hat{D}^\dagger \hat{D} \rangle$. However, there are many
situations where the mean-field approximation is not a valid
description of the physics. A prominent example is the Bose-Hubbard
model in 1D \cite{cazalilla2011, ejima2011, kuhner2000, pino2012,
  pino2013}. Observing the transition in 1D by light at fixed density
was considered to be difficult \cite{rogers2014} or even impossible
\cite{roth2003}. By contrast, here we propose varying the density or
chemical potential, which sharply identifies the transition. We
perform these calculations numerically by calculating the ground state
using DMRG methods \cite{tnt} from which we can compute all the
necessary atomic observables. Experiments typically use an additional
harmonic confining potential on top of the optical lattice to keep the
atoms in place which means that the chemical potential will vary in
space. However, with careful consideration of the full
($\mu/2J^\text{cl}$, $U/2J^\text{cl}$) phase diagrams in
Fig. \ref{fig:SFMI}(d,e) our analysis can still be applied to the
system \cite{batrouni2002}.

The 1D phase transition is best understood in terms of two-point
correlations as a function of their separation \cite{giamarchi}. In
the Mott insulating phase, the two-point correlations
$\langle \bd_i b_j \rangle$ and
$\langle \delta \hat{n}_i \delta \hat{n}_j \rangle$
($\delta \hat{n}_i =\hat{n}_i-\langle \hat{n}_i\rangle$) decay
exponentially with $|i-j|$. On the other hand the superfluid will
exhibit long-range order which in dimensions higher than one,
manifests itself with an infinite correlation length. However, in 1D
only pseudo long-range order happens and both the matter-field and
density fluctuation correlations decay algebraically \cite{giamarchi}.

The method we propose gives us direct access to the structure factor,
which is a function of the two-point correlation $\langle \delta
\hat{n}_i \delta \hat{n}_j \rangle$, by measuring the light
intensity. For two travelling waves maximally coupled to the density
(atoms are at light intensity maxima so $\hat{F} = \hat{D}$), the
quantum addition is given by
\begin{equation} 
  R =\sum_{i, j} \exp[i (\mathbf{k}_1 - \mathbf{k}_0)
  (\mathbf{r}_i - \mathbf{r}_j)] \langle \delta \hat{n}_i \delta
  \hat{n}_j \rangle,
\end{equation}

The angular dependence of $R$ for a Mott insulator and a superfluid is
shown in Fig. \ref{fig:SFMI}a, and there are two variables
distinguishing the states. Firstly, maximal $R$,
$R_\text{max} \propto \sum_i \langle \delta \hat{n}_i^2 \rangle$,
probes the fluctuations and compressibility $\kappa'$
($\langle \delta \hat{n}^2_i \rangle \propto \kappa' \langle \hat{n}_i
\rangle$).  The Mott insulator is incompressible and thus will have
very small on-site fluctuations and it will scatter little light
leading to a small $R_\text{max}$. The deeper the system is in the MI
phase (i.e. that larger the $U/2J^\text{cl}$ ratio is), the smaller
these values will be until ultimately it will scatter no light at all
in the $U \rightarrow \infty$ limit. In Fig. \ref{fig:SFMI}a this can
be seen in the value of the peak in $R$. The value $R_\text{max}$ in
the SF phase ($U/2J^\text{cl} = 0$) is larger than its value in the MI
phase ($U/2J^\text{cl} = 10$) by a factor of
$\sim$25. Figs. \ref{fig:SFMI}(b,d) show how the value of
$R_\text{max}$ changes across the phase transition. We see that the
transition shows up very sharply as $\mu$ is varied.

Secondly, being a Fourier transform, the width $W_R$ of the dip in $R$
is a direct measure of the correlation length $l$, $W_R \propto
1/l$. The Mott insulator being an insulating phase is characterised by
exponentially decaying correlations and as such it will have a very
large $W_R$. However, the superfluid in 1D exhibits pseudo long-range
order which manifests itself in algebraically decaying two-point
correlations \cite{giamarchi} which significantly reduces the dip in
the $R$. This can be seen in Fig. \ref{fig:SFMI}a and we can also see
that this identifies the phase transition very sharply as $\mu$ is
varied in Figs. \ref{fig:SFMI}(c,e). One possible concern with
experimentally measuring $W_R$ is that it might be obstructed by the
classical diffraction maxima which appear at angles corresponding to
the minima in $R$. However, the width of such a peak is much smaller
as its width is proportional to $1/M$.

It is also possible to analyse the phase transition quantitatively
using our method. Unlike in higher dimensions where an order parameter
can be easily defined within the MF approximation there is no such
quantity in 1D. However, a valid description of the relevant 1D low
energy physics is provided by Luttinger liquid theory
\cite{giamarchi}. In this model correlations in the supefluid phase as
well as the superfluid density itself are characterised by the
Tomonaga-Luttinger parameter, $K_b$. This parameter also identifies
the phase transition in the thermodynamic limit at $K_b = 1/2$. This
quantity can be extracted from various correlation functions and in
our case it can be extracted directly from $R$ \cite{ejima2011}. By
extracting this parameter from $R$ for various lattice lengths from
numerical DMRG calculations it was even possible to give a theoretical
estimate of the critical point for commensurate filling, $N = M$, in
the thermodynamic limit to occur at $U/2J^\text{cl} \approx 1.64$
\cite{ejima2011}. Our proposal provides a method to directly measure
$R$ in a lab which can then be used to experimentally determine the
location of the critical point in 1D.

So far both variables we considered, $R_\text{max}$ and $W_R$, provide
similar information. Next, we present a case where it is very
different. The Bose glass is a localized insulating phase with
exponentially decaying correlations but large compressibility and
on-site fluctuations in a disordered optical lattice. Therefore,
measuring both $R_\text{max}$ and $W_R$ will distinguish all the
phases. In a Bose glass we have finite compressibility, but
exponentially decaying correlations. This gives a large $R_\text{max}$
and a large $W_R$. A Mott insulator will also have exponentially
decaying correlations since it is an insulator, but it will be
incompressible. Thus, it will scatter light with a small
$R_\text{max}$ and large $W_R$. Finally, a superfluid will have long
range correlations and large compressibility which results in a large
$R_\text{max}$ and a small $W_R$.

\begin{figure}[htbp!]  
  \centering
  \includegraphics[width=\linewidth]{oph22}
  \caption[Mapping the Disoredered Phase Diagram]{The
    Mott-superfluid-glass phase diagrams for light scattering maximum
    $R_\text{max}/N_K$ (a) and width $W_R$ (b). Measurement of both
    quantities distinguish all three phases. Transition lines are
    shifted due to finite size effects \cite{roux2008}, but it is
    possible to apply well known numerical methods to extract these
    transition lines from such experimental data extracted from $R$
    \cite{ejima2011}. $K=M=N=35$.}
  \label{fig:BG}
\end{figure}

We confirm this in Fig. \ref{fig:BG} for simulations with the ratio of
superlattice- to trapping lattice-period $r\approx 0.77$ for various
disorder strengths $V$ \cite{roux2008}. Here, we only consider
calculations for a fixed density, because the usual interpretation of
the phase diagram in the ($\mu/2J^\text{cl}$, $U/2J^\text{cl}$) plane
for a fixed ratio $V/U$ becomes complicated due to the presence of
multiple compressible and incompressible phases between successive MI
lobes \cite{roux2008}. This way, we have limited our parameter space
to the three phases we are interested in: superfluid, Mott insulator,
and Bose glass. From Fig. \ref{fig:BG} we see that all three phases
can indeed be distinguished. In the 1D BHM there is no sharp MI-SF
phase transition in 1D at a fixed density \cite{cazalilla2011,
  ejima2011, kuhner2000, pino2012, pino2013} just like in
Figs. \ref{fig:SFMI}(d,e) if we follow the transition through the tip
of the lobe which corresponds to a line of unit density. However,
despite the lack of an easily distinguishable critical point it is
possible to quantitatively extract the location of the transition
lines by extracting the Tomonaga-Luttinger parameter from the
scattered light, $R$, in the same way it was done for an unperturbed
BHM \cite{ejima2011}.

Only recently \cite{derrico2014} a Bose glass phase was studied by
combined measurements of coherence, transport, and excitation spectra,
all of which are destructive techniques. Our method is simpler as it
only requires measurement of the quantity $R$ and additionally, it is
nondestructive.

\section{Conclusions}

In summary, we proposed a nondestructive method to probe quantum gases
in an optical lattice. Firstly, we showed that the density-term in
scattering has an angular distribution richer than classical
diffraction, derived generalized Bragg conditions, and estimated
parameters for the only two relevant experiments to date
\cite{weitenberg2011, miyake2011}. Secondly, we proposed how to
measure the matter-field interference by concentrating light between
the sites. This corresponds to interference at the shortest possible
distance in an optical lattice. By contrast, standard destructive
time-of-flight measurements deal with far-field interference and a
relatively near-field one was used in Ref. \cite{miyake2011}. This
defines most processes in optical lattices. E.g. matter-field phase
changes may happen not only due to external gradients, but also due to
intriguing effects such quantum jumps leading to phase flips at
neighbouring sites and sudden cancellation of tunneling
\cite{vukics2007}, which should be accessible by our method. In
mean-field, one can measure the matter-field amplitude (order
parameter), quadratures and squeezing. This can link atom optics to
areas where quantum optics has already made progress, e.g., quantum
imaging \cite{golubev2010, kolobov1999}, using an optical lattice as
an array of multimode nonclassical matter-field sources with a high
degree of entanglement for quantum information processing. Thirdly, we
demonstrated how the method accesses effects beyond mean-field and
distinguishes all the phases in the Mott-superfluid-glass transition,
which is currently a challenge \cite{derrico2014}. Based on
off-resonant scattering, and thus being insensitive to a detailed
atomic level structure, the method can be extended to molecules
\cite{LP2013}, spins, and fermions \cite{ruostekoski2009}.
