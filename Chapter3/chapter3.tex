%*******************************************************************************
%*********************************** Third Chapter *****************************
%*******************************************************************************

\chapter{Probing Correlations by Global 
Nondestructive Addressing} %Title of the Third Chapter

\ifpdf
    \graphicspath{{Chapter3/Figs/Raster/}{Chapter3/Figs/PDF/}{Chapter3/Figs/}}
\else
    \graphicspath{{Chapter3/Figs/Vector/}{Chapter3/Figs/}}
\fi


%********************************** %First Section  **************************************

\section{Introduction}

Having developed the basic theoretical framework within which we can
treat the fully quantum regime of light-matter interactions we now
consider possible applications. There are three prominent directions
in which we can proceed: nondestructive probing of the quantum state
of matter, quantum measurement backaction induced dynamics and quantum
optical lattices. Here, we deal with the first of the three options.

In this chapter we develop a method to measure properties of ultracold
gases in optical lattices by light scattering. In the previous chapter
we have shown that quantum light field couples to the bosons via the
operator $\hat{F}$. This is the key element of the scheme we propose
as this makes it sensitive to the quantum state of the matter and all
of its possible superpositions which will be reflected in the quantum
state of the light itself. We have also shown in section
\ref{sec:derivation} that this coupling consists of two parts, a
density component $\hat{D}$ given by Eq. \eqref{eq:D}, and a phase
component $\hat{B}$ given by Eq. \eqref{eq:B}. Therefore, when probing
the quantum state of the ultracold gas we can have access to not only
density correlations, but also matter-field interference at its
shortest possible distance in an optical lattice, i.e.~the lattice
period. Previous work on quantum non-demolition (QND) schemes
\cite{rogers2014, mekhov2007prl, eckert2008} probe only the density
component as it is generally challenging to couple to the matter-field
observables directly. Here, we will consider nondestructive probing of
both density and interference operators.

Firstly, we will consider the simpler and more typical case of
coupling to the atom number operators via $\hat{F} =
\hat{D}$. However, we show that light diffraction in this regime has
several nontrivial characteristics due to the fully quantum nature of
the interaction. Firstly, we show that the angular distribution has
multiple interesting features even when classical diffraction is
forbidden facilitating their experimental observation. We derive new
generalised Bragg diffraction conditions which are different to their
classical counterpart. Furthermore, due to the fully quantum nature of
the interaction our proposal is capable of probing the quantum state
beyond mean-field prediction. We demonstrate this by showing that this
scheme is capable of distinguishing all three phases in the Mott
insulator - superfluid - Bose glass phase transition in a 1D
disordered optical lattice which is not very well described by a
mean-field treatment. We underline that transitions in 1D are much
more visible when changing an atomic density rather than for
fixed-density scattering. It was only recently that an experiment
distinguished a Mott insulator from a Bose glass \cite{derrico2014}
via a series of destructive measurements. Our proposal, on the other
hand, is nondestructive and is capable of extracting all the relevant
information in a single experiment.

Having shown the possibilities created by this nondestructive
measurement scheme we move on to considering light scattering from the
phase related observables via the operator $\hat{F} = \hat{B}$. This
enables in-situ probing of the matter-field coherence at its shortest
possible distance in an optical lattice, i.e. the lattice period,
which defines key processes such as tunnelling, currents, phase
gradients, etc. This is in contrast to standard destructive
time-of-flight measurements which deal with far-field interference
although a relatively near-field scheme was use in
Ref. \cite{miyake2011}. We show how within the mean-field treatment,
this enables measurements of the order parameter, matter-field
quadratures and squeezing. This can have an impact on atom-wave
metrology and information processing in areas where quantum optics
already made progress, e.g., quantum imaging with pixellized sources
of non-classical light \cite{golubev2010, kolobov1999}, as an optical
lattice is a natural source of multimode nonclassical matter waves.

\section{Coupling to the Quantum State of Matter}

As we have seen in section \ref{sec:a} under certain approximations
the scattered light mode, $\a_1$, is linked to the quantum state of
matter via 
\begin{equation}
  \label{eq:a-3}
  \a_1 = C \hat{F} = C \left(\hat{D} + \hat{B} \right),
\end{equation}
where the atomic operators $\hat{D}$ and $\hat{B}$, given by
Eq. \eqref{eq:D} and Eq. \eqref{eq:B}, are responsible for the
coupling to on-site density and inter-site interference
respectively. It is crucial to note that light couples to the bosons
via an operator as this makes it sensitive to the quantum state of the
matter as this will imprint the fluctuations in the quantum state of
the scattered light.

Here, we will use this fact that the light is sensitive to the atomic
quantum state due to the coupling of the optical and matter fields via
operators in order to develop a method to probe the properties of an
ultracold gas. Therefore, we neglect the measurement back-action and
we will only consider expectation values of light observables. Since
the scheme is nondestructive (in some cases, it even satisfies the
stricter requirements for a QND measurement \cite{mekhov2012,
  mekhov2007pra}) and the measurement only weakly perturbs the system,
many consecutive measurements can be carried out with the same atoms
without preparing a new sample. We will show how the extreme
flexibility of the the measurement operator $\hat{F}$ allows us to
probe a variety of different atomic properties in-situ ranging from
density correlations to matter-field interference.

\subsection{On-site Density Measurements}

We have seen in section \ref{sec:B} that typically the dominant term
in $\hat{F}$ is the density term $\hat{D}$ \cite{LP2009,
  mekhov2007pra, rist2010, lakomy2009, ruostekoski2009}. This is
simply due to the fact that atoms are localised with lattice sites
leading to an effective coupling with atom number operators instead of
inter-site interference terms. Therefore, we will first consider
nondestructive probing of the density related observables of the
quantum gas. However, we will focus on the novel nontrivial aspects
that go beyond the work in Ref. \cite{mekhov2012, mekhov2007prl,
  mekhov2007pra} which only considered a few extremal cases.

As we are only interested in the quantum information imprinted in the
state of the optical field we will simplify our analysis by
considering the light scattering to be much faster than the atomic
tunnelling. Therefore, our scheme is actually a QND scheme
\cite{rogers2014, mekhov2007prl, mekhov2007pra, eckert2008} as
normally density-related measurements destroy the matter-phase
coherence since it is its conjugate variable, but here we neglect the
$\bd_i b_j$ terms. Furthermore, we will consider a deep
lattice. Therefore, the Wannier functions will be well localised
within their corresponding lattice sites and thus the coefficients
$J_{i,i}$ reduce to $u_1^*(\b{r}_i) u_0(\b{r}_i)$ leading to
\begin{equation}
  \label{eq:D-3}
  \hat{D}=\sum_i^K u_1^*(\b{r}_i) u_0(\b{r}_i) \hat{n}_i,
\end{equation} 
which for travelling
[$u_l(\b{r})=\exp(i \b{k}_l \cdot \b{r}+i\varphi_l)$] or standing
[$u_l(\b{r})=\cos(\b{k}_l \cdot \b{r}+\varphi_l)$] waves is just a
density Fourier transform at one or several wave vectors
$\pm(\b{k}_1 \pm \b{k}_0)$. 

We will now define a new auxiliary quantity to aid our analysis,
\begin{equation}
  \label{eq:R}
  R = \langle \ad_1 \a_1 \rangle - | \langle \a_1 \rangle |^2,
\end{equation}
which we will call the ``quantum addition'' to light scattering. By
construction $R$ is simply the full light intensity minus the
classical field diffraction. In order to justify its name we will show
that this quantity depends purely quantum mechanical properties of the
ultracold gase. We will substitute $\a_1 = C \hat{D}$ using
Eq. \eqref{eq:D-3} into our expression for $R$ in Eq. \eqref{eq:R} and
we will make use of the shorthand notation
$A_i = u_1^*(\b{r}_i) u_0(\b{r}_i)$. The result is
\begin{equation}
  R = |C|^2 \sum_{i,j}^K A^*_i A_j \langle \delta \hat{n}_i \delta
  \hat{n}_j \rangle,
\end{equation}
where $\delta \hat{n}_i = \hat{n}_i - \langle \hat{n}_i
\rangle$. Thus, we can clearly see that $R$ is a result of light
scattering from fluctuations in the atom number which is a purely
quantum mechanical property of a system. Therefore, $R$, the ``quantum
addition'' faithfully represents the new contribution from the quantum
light-matter interaction to the diffraction pattern.

If instead we are interested in quantities linear in $\hat{D}$, we can
measure the quadrature of the light fields which in section
\ref{sec:a} we saw that $\hat{X}_\phi = |C| \hat{X}^F_\beta$. For the
case when both the scattered mode and probe are travelling waves the
quadrature
\begin{equation} 
  \hat{X}^F_\beta = \frac{1}{2} \left( \hat{F} e^{-i \beta} +
    \hat{F}^\dagger e^{i \beta} \right) = \sum_i^K \hat{n}_i\cos[(\b{k}_1 - \b{k}_2) \cdot
  \b{r}_i - \beta].
\end{equation} 
Note that different light quadratures are differently coupled to the
atom distribution, hence by varying the local oscillator phase, and
thus effectively $\beta$, and/or the detection angle one can scan the
whole range of couplings. A similar expression exists for $\hat{D}$
for a standing wave probe, where $\beta$ is replaced by $\varphi_0$,
and scanning is achieved by varying the position of the wave with
respect to atoms.

The ``quantum addition'', $R$, and the quadrature variance, $(\Delta
X^F_\beta)^2$, are both quadratic in $\a_1$ and both rely heavily on
the quantum state of the matter. Therefore, they will have a
nontrivial angular dependence, showing more peaks than classical
diffraction. Furthermore, these peaks can be tuned very easily with
$\beta$ or $\varphi_l$. Fig. \ref{fig:scattering} shows the angular
dependence of $R$ for the case when the scattered mode is a standing
wave and the probe is a travelling wave scattering from bosons in a 3D
optical lattice. The first noticeable feature is the isotropic
background which does not exist in classical diffraction. This
background yields information about density fluctuations which,
according to mean-field estimates (i.e.~inter-site correlations are
ignored), are related by $R = K( \langle \hat{n}^2 \rangle - \langle
\hat{n} \rangle^2 )/2$. In Fig. \ref{fig:scattering} we can see a
significant signal of $R = |C|^2 N_K/2$, because it shows scattering
from an ideal superfluid which has significant density fluctuations
with correlations of infinte range. However, as the parameters of the
lattice are tuned across the phase transition into a Mott insulator
the signal goes to zero. This is because the Mott insulating phase has
well localised atoms at each site which suppresses density
fluctuations entirely leading to absolutely no ``quantum addition''.

\begin{figure}[htbp!]
  \centering
  \includegraphics[width=\linewidth]{Ep1}
  \caption[Light Scattering Angular Distribution]{Light intensity
    scattered into a standing wave mode from a superfluid in a 3D
    lattice (units of $R/(|C|^2N_K)$). Arrows denote incoming
    travelling wave probes. The classical Bragg condition,
    $\Delta \b{k} = \b{G}$, is not fulfilled, so there is no classical
    diffraction, but intensity still shows multiple peaks, whose
    heights are tunable by simple phase shifts of the optical beams:
    (a) $\varphi_1=0$; (b) $\varphi_1=\pi/2$. Interestingly, there is
    also a significant uniform background level of scattering which
    does not occur in its classical counterpart. }
  \label{fig:scattering}
\end{figure}

We can also observe maxima at several different angles in
Fig. \ref{fig:scattering}. Interestingly, they occur at different
angles than predicted by the classical Bragg condition. Moreover, the
classical Bragg condition is actually not satisfied which means there
actually is no classical diffraction on top of the ``quantum
addition'' shown here. Therefore, these features would be easy to see
in an experiment as they wouldn't be masked by a stronger classical
signal.  We can even derive the generalised Bragg conditions for the
peaks that we can see in Fig. \ref{fig:scattering}. 

\mynote{Derive and show these Bragg conditions below}

As $(\Delta X^F_\beta)^2$ and $R$ are quadratic variables,
the generalized Bragg conditions for the peaks are
$2 \Delta \b{k} = \b{G}$ for quadratures of travelling waves, where
$\Delta \b{k} = \b{k}_0 - \b{k}_1$ and $\b{G}$ is the reciprocal
lattice vector, and $2 \b{k}_1 = \b{G}$ for standing wave $\a_1$ and
travelling $\a_0$, which is clearly different from the classical Bragg
condition $\Delta \b{k} = \b{G}$. The peak height is tunable by the
local oscillator phase or standing wave shift as seen in Fig.
\ref{fig:scattering}b.

A quantum signal that isn't masked by classical diffraction is very
useful for future experimental realisability. However, it is still
unclear whether this signal would be strong enough to be
visible. After all, a classical signal scales as $N_K^2$ whereas here
we have only seen a scaling of $N_K$. In section \ref{sec:Efield} we
have estimated the mean photon scattering rates integrated over the
solid angle for the only two experiments so far on light diffraction
from truly ultracold bosons where the measurement object was light
\begin{equation} 
  n_{\Phi}= \left(\frac{\Omega_0}{\Delta_a}\right)^2 \frac{\Gamma K}{8}
  (\langle\hat{n}^2\rangle-\langle\hat{n}\rangle^2).
\end{equation} 
These results can be applied directly to the scattering patters in
Fig. \ref{fig:scattering}. Therefore, the background signal should
reach $n_\Phi \approx 10^6$ s$^{-1}$ in Ref. \cite{weitenberg2011}
(150 atoms in 2D), and $n_\Phi \approx 10^{11}$ s$^{-1}$ in
Ref. \cite{miyake2011} ($10^5$ atoms in 3D). These numbers show that
the diffraction patterns we have seen due to the ``quantum addition''
should be visible using currently available technology, especially
since the most prominent features, such as Bragg diffraction peaks, do
not coincide at all with the classical diffraction pattern.

\subsection{Mapping the quantum phase diagram}

We have shown that scattering from atom number operators leads to a
purely quantum diffraction pattern which depends on the density
fluctuations and their correlations. We have also seen that this
signal should be strong enough to be visible using currently available
technology. However, so far we have not looked at what this can tell
us about the quantum state of matter. We have briefly mentioned that a
deep superfluid will scatter a lot of light due to its infinite range
correlations and a Mott insulator will not contriute any ``quantum
addition'' at all, but we have not look at the quantum phase
transition between these two phases. In two or higher dimensions this
has a rather simple answer as the Bose-Hubbard phase transition is
described well by mean-field theories and it has a sharp transition at
the critical point. This means that the ``quantum addition'' signal
would drop rapidly at the critical point and go to zero as soon as it
was crossed. However, $R$ given by Eq. \eqref{eq:R} clearly contains
much more information.

There are many situations where the mean-field approximation is not a
valid description of the physics. A prominent example is the
Bose-Hubbard model in 1D \cite{cazalilla2011, ejima2011, kuhner2000,
  pino2012, pino2013}. Observing the transition in 1D by light at
fixed density was considered to be difficult \cite{rogers2014} or even
impossible \cite{roth2003}. This is, because the one-dimensional
quantum phase transition is in a different universality class than its
higher dimensional counterparts. The energy gap, which is the order
parameter, decays exponentially slowly across the phase transition
making it difficult to identify the phase transition even in numerical
simulations. Here, we will show the avaialable tools provided by the
``quantum addition'' that allows one to nondestructively map this
phase transition and distinguish the superfluid and Mott insulator
phases.

The 1D phase transition is best understood in terms of two-point
correlations as a function of their separation \cite{giamarchi}. In
the Mott insulating phase, the two-point correlations $\langle \bd_i
b_j \rangle$ and $\langle \delta \hat{n}_i \delta \hat{n}_j \rangle$
($\delta \hat{n}_i =\hat{n}_i-\langle \hat{n}_i\rangle$) decay
exponentially with $|i-j|$. This is a characteristic of insulators. On
the other hand the superfluid will exhibit long-range order which in
dimensions higher than one, manifests itself with an infinite
correlation length. However, in 1D only pseudo long-range order
happens and both the matter-field and density fluctuation correlations
decay algebraically \cite{giamarchi}.

The method we propose gives us direct access to the structure factor,
which is a function of the two-point correlation $\langle \delta
\hat{n}_i \delta \hat{n}_j \rangle$. This quantity can be extracted
from the measured light intensity bu considering the ``quantum
addition''. We will consider the case when both the probe and
scattered modes are plane waves which can be easily achieved in free
space. We will again consider the case of light being maximally
coupled to the density ($\hat{F} = \hat{D}$). Therefore, the quantum
addition is given by
\begin{equation} 
  R =\sum_{i, j} \exp[i (\mathbf{k}_1 - \mathbf{k}_0)
  (\mathbf{r}_i - \mathbf{r}_j)] \langle \delta \hat{n}_i \delta
  \hat{n}_j \rangle.
\end{equation}

\mynote{can put in more detail here with equations}

This alone allows us to analyse the phase transition quantitatively
using our method. Unlike in higher dimensions where an order parameter
can be easily defined within the mean-field approximation as a simple
expectation value of some quantity, the situation in 1D is more
complex as it is difficult to directly access the excitation energy
gap which defines this phase transition. However, a valid description
of the relevant 1D low energy physics is provided by Luttinger liquid
theory \cite{giamarchi}. In this model correlations in the supefluid
phase as well as the superfluid density itself are characterised by
the Tomonaga-Luttinger parameter, $K_b$. This parameter also
identifies the critical point in the thermodynamic limit at $K_b =
1/2$. This quantity can be extracted from various correlation
functions and in our case it can be extracted directly from $R$
\cite{ejima2011}. This quantity was used in numerical calculations
that used highly efficient density matrix renormalisation group (DMRG)
methods to calculate the ground state to subsequently fit the
Luttinger theoru to extract this parameter $K_b$. These calculations
yield a theoretical estimate of the critical point in the
thermodynamic limit for commensurate filling in 1D to be at
$U/2J^\text{cl} \approx 1.64$ \cite{ejima2011}. Our proposal provides
a method to directly measure $R$ nondestructively in a lab which can
then be used to experimentally determine the location of the critical
point in 1D.

However, whilst such an approach will yield valuable quantitative
results we will instead focus on its qualitative features which give a
more intuitive understanding of what information can be extracted from
$R$. This is because the superfluid to Mott insulator phase transition
is well understood, so there is no reason to dwell on its quantitative
aspects. However, our method is much more general than the
Bose-Hubbard model as it can be easily applied to many other systems
such as fermions, photonic circuits, optical lattices qith quantum
potentials, etc. Therefore, by providing a better physical picture of
what information is carried by the ``quantum addition'' it should be
easier to see its usefuleness in a broader context.

We calculate the phase diagram of the Bose-Hubbard Hamiltonian given
by
\begin{equation}
    \hat{H}_\mathrm{dis} = -J^\mathrm{cl} \sum_{\langle i, j \rangle}
    \bd_i b_j + \frac{U}{2} \sum_i \hat{n}_i (\hat{n}_i - 1) - \mu
    \sum_i \hat{n}_i,
\end{equation}
where the $\mu$ is the chemical potential. We have introduced the last
term as we are interested in grand canonical ensemble calculations as
we want to see how the system's behaviour changes as density is
varied. We perform numerical calculations of the ground state using
DMRG methods \cite{tnt} from which we can compute all the necessary
atomic observables. Experiments typically use an additional harmonic
confining potential on top of the optical lattice to keep the atoms in
place which means that the chemical potential will vary in
space. However, with careful consideration of the full
($\mu/2J^\text{cl}$, $U/2J^\text{cl}$) phase diagrams in
Fig. \ref{fig:SFMI}(d,e) our analysis can still be applied to the
system \cite{batrouni2002}.

\begin{figure}[htbp!]  
  \centering
  \includegraphics[width=\linewidth]{oph11}
  \caption[Mapping the Bose-Hubbard Phase Diagram]{(a) The angular
    dependence of scattered light $R$ for a superfluid (thin black,
    left scale, $U/2J^\text{cl} = 0$) and Mott insulator (thick green,
    right scale, $U/2J^\text{cl} =10$). The two phases differ in both
    their value of $R_\text{max}$ as well as $W_R$ showing that
    density correlations in the two phases differ in magnitude as well
    as extent. Light scattering maximum $R_\text{max}$ is shown in (b,
    d) and the width $W_R$ in (c, e).  It is very clear that varying
    chemical potential $\mu$ or density $\langle n\rangle$ sharply
    identifies the superfluid-Mott insulator transition in both
    quantities. (b) and (c) are cross-sections of the phase diagrams
    (d) and (e) at $U/2J^\text{cl}=2$ (thick blue), 3 (thin purple),
    and 4 (dashed blue). Insets show density dependencies for the
    $U/(2 J^\text{cl}) = 3$ line. $K=M=N=25$.}
	\label{fig:SFMI}
\end{figure}

We then consider probing these ground states using our optical scheme
and we calculate the ``quantum addition'', $R$. The angular dependence
of $R$ for a Mott insulator and a superfluid is shown in
Fig. \ref{fig:SFMI}a, and we note that there are two variables
distinguishing the states. Firstly, maximal $R$, $R_\text{max} \propto
\sum_i \langle \delta \hat{n}_i^2 \rangle$, probes the fluctuations
and compressibility $\kappa'$ ($\langle \delta \hat{n}^2_i \rangle
\propto \kappa' \langle \hat{n}_i \rangle$).  The Mott insulator is
incompressible and thus will have very small on-site fluctuations and
it will scatter little light leading to a small $R_\text{max}$. The
deeper the system is in the insulating phase (i.e. that larger the
$U/2J^\text{cl}$ ratio is), the smaller these values will be until
ultimately it will scatter no light at all in the $U \rightarrow
\infty$ limit. In Fig. \ref{fig:SFMI}a this can be seen in the value
of the peak in $R$. The value $R_\text{max}$ in the superfluid phase
($U/2J^\text{cl} = 0$) is larger than its value in the Mott insulating
phase ($U/2J^\text{cl} = 10$) by a factor of
$\sim$25. Figs. \ref{fig:SFMI}(b,d) show how the value of
$R_\text{max}$ changes across the phase transition. There are a few
things to note at this point. Firstly, if we follow the transition
along the line corresponding to commensurate filling (i.e.~any line
that is in between the two white lines in Fig. \ref{fig:SFMI}d) we see
that the transition is very smooth and it is hard to see a definite
critical point. This is due to the energy gap closing exponentially
slowly which makes precise identification of the critical point
extremely difficult. The best option at this point would be to fit
Tomonage-Luttinger theory to the results in order to find this
critical point. However, we note that there is a drastic change in
signal as the chemical potential (and thus the density) is
varied. This is highlighted in Fig. \ref{fig:SFMI}b which shows how
the Mott insulator can be easily identified by a dip in the quantity
$R_\text{max}$.

Secondly, being a Fourier transform, the width $W_R$ of the dip in $R$
is a direct measure of the correlation length $l$, $W_R \propto
1/l$. The Mott insulator being an insulating phase is characterised by
exponentially decaying correlations and as such it will have a very
large $W_R$. On the other hand, the superfluid in 1D exhibits pseudo
long-range order which manifests itself in algebraically decaying
two-point correlations \cite{giamarchi} which significantly reduces
the dip in the $R$. This can be seen in
Fig. \ref{fig:SFMI}a. Furthermore, just like for $R_\text{max}$ we see
that the transition is much sharper as $\mu$ is varied. This is shown
in Figs. \ref{fig:SFMI}(c,e). Notably, the difference in angle between
a superfluid and an insulating state is fairly significant $\sim
20^\circ$ which should make the two phases easy to identify using this
measure. In this particular case, measuring $W_R$ in the Mott phase is
not very practical as the insulating phase does not scatter light
(small $R_\mathrm{max}$). The phase transition information is easier
extracted from $R_\mathrm{max}$. However, this is not always the case
and we will shortly see how certain phases of matter scatter a lot of
light and can be distinguished using measurements of $W_R$. Another
possible concern with experimentally measuring $W_R$ is that it might
be obstructed by the classical diffraction maxima which appear at
angles corresponding to the minima in $R$. However, the width of such
a peak is much smaller as its width is proportional to $1/M$.

So far both variables we considered, $R_\text{max}$ and $W_R$, provide
similar information. They both take on values at one of its extremes
in the Mott insulating phase and they change drastically across the
phase transition into the superfluid phase. Next, we present a case
where it is very different. We will again consider ultracold bosons in
an optical lattice, but this time we introduce some disorder. We do
this by adding an additional periodic potential on top of the exisitng
setup that is incommensurate with the original lattice. The resulting
Hamiltonian can be shown to be
\begin{equation}
  \hat{H}_\mathrm{dis} = -J^\mathrm{cl} \sum_{\langle i, j \rangle}
  \bd_i b_j + \frac{U}{2} \sum_i \hat{n}_i (\hat{n}_i - 1) +
  \frac{V}{2} \sum_i \left[ 1 + \cos (2 r \pi m + 2 \phi) \right]
  \hat{n}_i,
\end{equation}
where $V$ is the strength of the superlattice potential, $r$ is the
ratio of the superlattice and trapping wave vectors and $\phi$ is some
phase shift between the two lattice potentials \cite{roux2008}. The
first two terms are the standard Bose-Hubbard Hamiltonian. The only
modification is an additionally spatially varying potential shift. We
will only consider the phase diagram at fixed density as the
introduction of disorder makes the usual interpretation of the phase
diagram in the ($\mu/2J^\text{cl}$, $U/2J^\text{cl}$) plane for a
fixed ratio $V/U$ complicated due to the presence of multiple
compressible and incompressible phases between successive MI lobes
\cite{roux2008}. Therefore, the chemical potential no longer appears
in the Hamiltonian as we are no longer considering the grand canonical
ensemble.

The reason for considering such a system is that it introduces a
third, competing phase, the Bose glass into our phase diagram. It is
an insulating phase like the Mott insulator, but it has local
superfluid susceptibility making it compressible. Therefore this
localized insulating phase will have exponentially decaying
correlations just like the Mott phase, but it will have large on-site
fluctuations just like the compressible superfluid phase. As these are
the two physical variables encoded in $R$ measuring both
$R_\text{max}$ and $W_R$ will provide us with enough information to
distinguish all three phases. In a Bose glass we have finite
compressibility, but exponentially decaying correlations. This gives a
large $R_\text{max}$ and a large $W_R$. A Mott insulator will also
have exponentially decaying correlations since it is an insulator, but
it will be incompressible. Thus, it will scatter light with a small
$R_\text{max}$ and large $W_R$. Finally, a superfluid will have long
range correlations and large compressibility which results in a large
$R_\text{max}$ and a small $W_R$.

\begin{figure}[htbp!]  
  \centering
  \includegraphics[width=\linewidth]{oph22}
  \caption[Mapping the Disoredered Phase Diagram]{The
    Mott-superfluid-glass phase diagrams for light scattering maximum
    $R_\text{max}/N_K$ (a) and width $W_R$ (b). Measurement of both
    quantities distinguish all three phases. Transition lines are
    shifted due to finite size effects \cite{roux2008}, but it is
    possible to apply well known numerical methods to extract these
    transition lines from such experimental data extracted from $R$
    \cite{ejima2011}. $K=M=N=35$.}
  \label{fig:BG}
\end{figure}

We confirm this in Fig. \ref{fig:BG} for simulations with the ratio of
superlattice- to trapping lattice-period $r\approx 0.77$ for various
disorder strengths $V$ \cite{roux2008}. From Fig. \ref{fig:BG} we see
that all three phases can indeed be distinguished. From the
$R_\mathrm{max}$ plot we can distinguish the two compressible phases,
the superfluid and Bose glass, from the incompressible phase, the Mott
insulator. We can now also distinguish the Bose glass phase from the
superfluid, because only one of them is an insulator and thus the
angular width of their scattering patterns will reveal this
information. However, unlike the Mott insulator, the Bose glass
scatters a lot of light (large $R_\mathrm{max}$) enabling such
measurements. Since we performed these calculations at a fixed density
the Mott to superfluid phase transition is not particularly sharp
\cite{cazalilla2011, ejima2011, kuhner2000, pino2012, pino2013} just
like we have seen in Figs. \ref{fig:SFMI}(d,e) if we follow the
transition through the tip of the lobe which corresponds to a line of
unit density. However, despite the lack of an easily distinguishable
critical point, as we have already discussed, it is possible to
quantitatively extract the location of the transition lines by
extracting the Tomonaga-Luttinger parameter from the scattered light,
$R$, in the same way it was done for an unperturbed BHM
\cite{ejima2011}.

Only recently \cite{derrico2014} a Bose glass phase was studied by
combined measurements of coherence, transport, and excitation spectra,
all of which are destructive techniques. Our method is simpler as it
only requires measurement of the quantity $R$ and additionally, it is
nondestructive.

\subsection{Matter-field interference measurements}

We now focus on enhancing the interference term $\hat{B}$ in the
operator $\hat{F}$. 

Firstly, we will use this result to show how one can probe
$\langle \hat{B} \rangle$ which in MF gives information about the
matter-field amplitude, $\Phi = \langle b \rangle$. 

Hence, by measuring the light quadrature we probe the kinetic energy
and, in MF, the matter-field amplitude (order parameter) $\Phi$:
$\langle \hat{X}^F_{\beta=0} \rangle = | \Phi |^2
\mathcal{F}[W_1](2\pi/d) (K-1)$.

Secondly, we show that it is also possible to access the fluctuations
of matter-field quadratures $\hat{X}^b_\alpha = (b e^{-i\alpha} + \bd
e^{i\alpha})/2$, which in MF can be probed by measuring the variance
of $\hat{B}$. Across the phase transition, the matter field changes
its state from Fock (in MI) to coherent (deep SF) through an
amplitude-squeezed state as shown in Fig. \ref{Quads}(a,b). 

Assuming $\Phi$ is real in MF:
\begin{equation}
  \label{intensity} 
  \langle \ad_1 \a_1 \rangle = 2 |C|^2(K-1)\mathcal{F}^2[W_1](\frac{\pi}{d})
  \times [ ( \langle b^2 \rangle - \Phi^2 )^2 + ( n - \Phi^2 ) ( 1 +n - \Phi^2 ) ]
\end{equation} 
and it is shown as a function of $U/(zJ^\text{cl})$ in
Fig. \ref{Quads}. Thus, since measurement in the diffraction maximum
yields $\Phi^2$ we can deduce $\langle b^2 \rangle - \Phi^2$ from the
intensity. This quantity is of great interest as it gives us access to
the quadrature variances of the matter-field
\begin{equation} 
  (\Delta X^b_{0,\pi/2})^2 = 1/4 + [(n - \Phi^2) \pm
  (\langle b^2 \rangle - \Phi^2)]/2,
\end{equation} 
where $n=\langle\hat{n}\rangle$ is the mean on-site atomic density.

\begin{figure}[htbp!]
  \centering
  \includegraphics[width=\linewidth]{Quads}
  \captionsetup{justification=centerlast,font=small}
  \caption[Mean-Field Matter Quadratures]{Photon number scattered in a
    diffraction minimum, given by Eq. (\ref{intensity}), where
    $\tilde{C} = 2 |C|^2 (K-1) \mathcal{F}^2 [W_1](\pi/d)$.  More
    light is scattered from a MI than a SF due to the large
    uncertainty in phase in the insulator. (a) The variances of
    quadratures $\Delta X^b_0$ (solid) and $\Delta X^b_{\pi/2}$
    (dashed) of the matter field across the phase transition. Level
    1/4 is the minimal (Heisenberg) uncertainty. There are three
    important points along the phase transition: the coherent state
    (SF) at A, the amplitude-squeezed state at B, and the Fock state
    (MI) at C. (b) The uncertainties plotted in phase space.}
	\label{Quads}
\end{figure}

Probing $\hat{B}^2$ gives us access to kinetic energy fluctuations
with 4-point correlations ($\bd_i b_j$ combined in pairs). Measuring
the photon number variance, which is standard in quantum optics, will
lead up to 8-point correlations similar to 4-point density
correlations \cite{mekhov2007pra}. These are of significant interest,
because it has been shown that there are quantum entangled states that
manifest themselves only in high-order correlations
\cite{kaszlikowski2008}.

Surprisingly, inter-site terms scatter more light from a Mott
insulator than a superfluid Eq. \eqref{intensity}, as shown in
Fig. \eqref{Quads}, although the mean inter-site density
$\langle \hat{n}(\b{r})\rangle $ is tiny in a MI. This reflects a
fundamental effect of the boson interference in Fock states. It indeed
happens between two sites, but as the phase is uncertain, it results
in the large variance of $\hat{n}(\b{r})$ captured by light as shown
in Eq. \eqref{intensity}. The interference between two macroscopic
BECs has been observed and studied theoretically
\cite{horak1999}. When two BECs in Fock states interfere a phase
difference is established between them and an interference pattern is
observed which disappears when the results are averaged over a large
number of experimental realizations. This reflects the large
shot-to-shot phase fluctuations corresponding to a large inter-site
variance of $\hat{n}(\b{r})$. By contrast, our method enables the
observation of such phase uncertainty in a Fock state directly between
lattice sites on the microscopic scale in-situ.

\section{Conclusions}

In this chapter we explored the possibility of nondestructively
probing a quantum gas trapped in an optical lattice using quantized
light. Firstly, we showed that the density-term in scattering has an
angular distribution richer than classical diffraction, derived
generalized Bragg conditions, and estimated parameters for two
relevant experiments \cite{weitenberg2011, miyake2011}. Secondly, we
demonstrated how the method accesses effects beyond mean-field and
distinguishes all the phases in the Mott-superfluid-glass transition,
which is currently a challenge \cite{derrico2014}. Finally, we looked
at measuring the matter-field interference via the operator $\hat{B}$
by concentrating light between the sites. This corresponds to probing
interference at the shortest possible distance in an optical
lattice. This is in contrast to standard destructive time-of-flight
measurements which deal with far-field interference and a relatively
near-field one was used in Ref. \cite{miyake2011}. This defines most
processes in optical lattices. E.g. matter-field phase changes may
happen not only due to external gradients, but also due to intriguing
effects such quantum jumps leading to phase flips at neighbouring
sites and sudden cancellation of tunneling \cite{vukics2007}, which
should be accessible by our method. We showed how in mean-field, one
can measure the matter-field amplitude (order parameter), quadratures
and squeezing. This can link atom optics to areas where quantum optics
has already made progress, e.g., quantum imaging \cite{golubev2010,
  kolobov1999}, using an optical lattice as an array of multimode
nonclassical matter-field sources with a high degree of entanglement
for quantum information processing. Since our scheme is based on
off-resonant scattering, and thus being insensitive to a detailed
atomic level structure, the method can be extended to molecules
\cite{LP2013}, spins, and fermions \cite{ruostekoski2009}.

