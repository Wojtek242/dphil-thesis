%*******************************************************************************
%*********************************** Second Chapter ****************************
%*******************************************************************************

%Title of the Second Chapter
\chapter{Quantum Optics of Quantum Gases}  

\ifpdf
    \graphicspath{{Chapter2/Figs/Raster/}{Chapter2/Figs/PDF/}{Chapter2/Figs/}}
\else
    \graphicspath{{Chapter2/Figs/Vector/}{Chapter2/Figs/}}
\fi


%********************************** % First Section  ****************************

\section{Ultracold Atoms in Optical Lattices}

%********************************** % Second Section  ***************************

\section{Quantum Optics of Quantum Gases}

Having introduced and described the behaviour of ultracold bosons
trapped and manipulated using classical light, it is time to extend
the discussion to quantized optical fields. We will first derive a
general Hamiltonian that describes the coupling of atoms with
far-detuned optical beams \cite{mekhov2012}. This will serve as the
basis from which we explore the system in different parameter regimes,
such as nondestructive measurement in free space or quantum
measurement backaction in a cavity.

We consider $N$ two-level atoms in an optical lattice with $M$
sites. For simplicity we will restrict our attention to spinless
bosons, although it is straightforward to generalise to fermions,
which yields its own set of interesting quantum phenomena
\cite{atoms2015, mazzucchi2016, mazzucchi2016af}, and other spin
particles. The theory can be also be generalised to continuous
systems, but the restriction to optical lattices is convenient for a
variety of reasons. Firstly, it allows us to precisely describe a
many-body atomic state over a broad range of parameter values due to
the inherent tunability of such lattices. Furthermore, this model is
capable of describing a range of different experimental setups ranging
from a small number of sites with a large filling factor (e.g.~BECs
trapped in a double-well potential) to a an extended multi-site
lattice with a low filling factor (e.g.~a system with one atom per
site will exhibit the Mott insulator to superfluid quantum phase
transition). \mynote{extra fermion citations, Piazza?}



