%*******************************************************************************
%*********************************** Fourth Chapter *****************************
%*******************************************************************************

\chapter{Quantum Measurement Backaction}  
% Title of the Fourth Chapter

\ifpdf
    \graphicspath{{Chapter4/Figs/Raster/}{Chapter4/Figs/PDF/}{Chapter4/Figs/}}
\else
    \graphicspath{{Chapter4/Figs/Vector/}{Chapter4/Figs/}}
\fi


%********************************** %First Section  **************************************

\section{Introduction}

This thesis is entirely concerned with the question of measuring a
quantum many-body system using quantized light. However, so far we
have only looked at expectation values in a nondestructive context
where we neglect the effect of the quantum wavefunction collapse. We
have shown that light provides information about various statistical
quantities of the quantum states of the atoms such as their
correlation functions. In general, any quantum measurement affects the
system even if it doesn't physically destroy it. In our model both
optical and matter fields are quantized and their interaction leads to
entanglement between the two subsystems. When a photon is detected and
the electromagnetic wavefunction of the optical field collapses, the
matter state is also affected due to this entanglement resulting in
quantum measurement backaction. Therefore, in order to determine these
quantities multiple measurements have to be performed to establish a
precise measurement of the expectation value which will require
repeated preparations of the initial state.

In the following chapters, we consider a different approach to quantum
measurement in open systems and instead of considering expectation
values we look at a single experimental run and the resulting dynamics
due to measurement backaction. Previously we were mostly interested in
extracting information about the quantum state of the atoms from the
scattered light. The flexibility in the measurement model was used to
enable probing of as many different quantum properties of the
ultracold gas as possible. By focusing on measurement backaction we
instead investigate the effect of photodetections on the dynamics of
the many-body gas as well as the possible quantum states that we can
prepare instead of what information can be extracted.

In this chapter, we introduce the necessary theory of quantum
measurement and backaction in open systems in order to lay a
foundation for the material that follows in which we apply these
concepts to a bosonic quantum gas. We first introduce the concept of
quantum trajectories which represent a single continuous series of
photon detections. We also present an alternative approach to open
systems in which the measurement outcomes are discarded as this will
be useful when trying to learn about dynamical features common to
every trajectory. In this case we use the density matrix formalism
which obeys the master equation. This approach is more common in
dissipative systems and we will highlight the differences between
these two different types of open systems. We conclude this chapter
with a new concept that will be central to all subsequent
discussions. In our model measurement is global, it couples to
operators that correspond to global properties of the quantum gas
rather than single-site quantities. This enables the possibility of
performing measurements that cannot distinguish certain sites from
each other. Due to a lack of ``which-way'' information this leads to
the creation of spatially nontrivial virtual lattices on top of the
physical lattice. This turns out to have significant consequences on
the dynamics of the system.

\section{Quantum Trajectories}

A simple intuitive concept of a quantum trajectory is that it is the
path taken by a quantum state over time during a single experimental
realisation. In particular, we consider states conditioned upon
measurement results such as the photodetection times. Such a
trajectory is generally stochastic in nature as light scattering is
not a deterministic process. Furthermore, they are in general
discotinuous as each detection event brings about a drastic change in
the quantum state due to the wavefunction collapse of the light field.

Before we discuss specifics relevant to our model of quantized light
interacting with a quantum gas we present a more general overview
which will be useful as some of the results in the following chapters
are more general. Measurement always consists of at least two
competing processes, two possible outcomes. If there is no competition
and only one outcome is possible then our measurement is meaningless
as it does not reveal any information about the system. In its
simplest form measurement consists of a series of events, such as the
detections of photons. Even though, on an intuitive level it seems
that we have defined only a single outcome, the event, this
arrangement actually consists of two mutually exclusive outcomes. At
any point in time an event either happens or it does not, a photon is
either detected or the detector remains silent, also known as a null
result. Both outcomes reveal some information about the system we are
investigating. For example, let us consider measuring the number of
atoms by measuring the number of photons they scatter. Each atom will
on average scatter a certain number of photons contributing to the
detection rate we observe. Therefore, if we record multiple photons at
a high rate we learn that the illuminated region must contain many
atoms. On the other hand, if there are few atoms to scatter the light
we will observe few detection events which we interpret as a
continuous series of non-detection events interspersed with the
occasional detector click. This trajectory informs us that there are
much fewer atoms being illuminated than previously.

Basic quantum mechanics tells us that such measurements will in
general affect the quantum state in some way. Each event will cause a
discontinuous quantum jump in the wavefunction of the system and it
will have a jump operator, $\c$, associated with it. The effect of an
event on the quantum state is simply the result of applying this jump
operator to the wavefunction, $| \psi (t) \rangle$,
\begin{equation}
  | \psi(t + \mathrm{d}t) \rangle = \frac{\c | \psi(t) \rangle}
  {\sqrt{\langle \cd \c \rangle (t)}},
\end{equation}
where the denominator is simply a normalising factor. The exact form
of the jump operator $\c$ will depend on the nature of the measurement
we are considering. For example, if we consider measuring the photons
escaping from a leaky cavity then $\c = \sqrt{2 \kappa} \hat{a}$,
where $\kappa$ is the cavity decay rate and $\hat{a}$ is the
annihilation operator of a photon in the cavity field. The null
measurement outcome has to be treated differently as it does not occur
at discrete time points like the detection events themselves. Its
effect is accounted for by a modification to the isolated Hamiltonian,
$\hat{H}_0$, time evolution in the form
\begin{equation}
  | \psi (t + \mathrm{d}t) \rangle = \left\{ \hat{1} - \mathrm{d}t
    \left[ i \hat{H}_0 + \frac{\cd \c}{2} - \frac{\langle \cd \c
        \rangle (t)}{2} \right] \right\} | \psi (t) \rangle.
\end{equation}
The effect of both outcomes can be included in a single stochastic
Schr\"{o}dinger equation given by
\begin{equation}
  \label{eq:SSE}
  \mathrm{d} | \psi(t) \rangle = \left[ \mathrm{d} N(t) \left(
      \frac{\c} {\sqrt{ \langle \cd \c \rangle (t)}} - \hat{1} \right)
    + \mathrm{d} t \left( \frac{\langle \cd \c \rangle (t)}{2} -
      \frac{ \cd \c}{2} - i \hat{H} \right) \right] | \psi(t) \rangle,
\end{equation}
where $\mathrm{d}N(t)$ is the stochastic increment to the number of
photodetections up to time $t$ which is equal to $1$ whenever a
quantum jump occurs and $0$ otherwise. Note that this equation has a
straightforward generalisation to multiple jump operators which we do
not consider here at all.

All trajectories that we calculate in the follwing chapters are
described by the stochastic Schr\"{o}dinger equation in
Eq. \eqref{eq:SSE}. The most straightforward way to solve it is to
replace the differentials by small time-setps $\delta t$. Then we
generate a random number $R(t)$ at every time-step and a jump is
applied, i.e.~$\mathrm{d}N(t) = 1$, if 
\begin{equation}
  R(t) < \langle \cd \c \rangle (t) \delta t.
\end{equation}

\section{The Master Equation}

\section{Global Measurement and ``Which-Way'' Information}