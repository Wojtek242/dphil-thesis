%*******************************************************************************
%*********************************** First Chapter *****************************
%*******************************************************************************

\chapter{Introduction}  %Title of the First Chapter

\ifpdf
    \graphicspath{{Chapter1/Figs/Raster/}{Chapter1/Figs/PDF/}{Chapter1/Figs/}}
\else
    \graphicspath{{Chapter1/Figs/Vector/}{Chapter1/Figs/}}
\fi


The field of ultracold gases has been rapidly growing ever since the
first Bose-Einstein condensate (BEC) was obtained in 1995
\cite{anderson1995, bradley1995, davis1995}. This new quantum state of
matter is characterised by a macroscopic occupancy of the single
particle ground state at which point the whole system behaves like a
single many-body quantum object \cite{PitaevskiiStringari}. This was
revolutionary as it enabled the study of coherent properties of
macroscopic systems rather than single atoms or photons. Furthermore,
the advanced state of laser cooling and manipulation technologies
meant that the degree of control and isolation from the environment
was far greater than was possible in condensed matter systems
\cite{lewenstein2007, bloch2008}. Initially, the main focus of the
research was on the properties of coherent matter waves, such as
interference properties \cite{andrews1997}, long-range phase coherence
\cite{bloch2000}, or quantised vortices \cite{matthews1999,
  madison2000, abo2001}. Fermi degeneracy in ultracold gases was
obtained shortly afterwards opening a similar field for fermions
\cite{demarco1999, schreck2001, truscott2001}.

In 1998 it was shown that a degenerate ultracold gas trapped in an
optical lattice is a near-perfect realisation of the Bose-Hubbard
model \cite{jaksch1998} and in 2002 it was already demonstrated in a
ground-breaking experiment \cite{greiner2002}. The Bose-Hubbard
Hamiltonian was previously known in the field of condensed matter
where it was considered a simple toy model. Despite its simplicity the
model exhibits a variety of different interesting phenomena such as
the quantum phase transition from a delocalised superfluid state to a
Mott insulator as the on-site interaction is increased above a
critical point which was originally studied in the context of liquid
helium \cite{fisher1989}. In contrast to a thermodynamic phase
transition, a quantum phase transition is driven by quantum
fluctuations and can occur at zero temperature. The ability to achieve
a Bose-Hubbard Hamiltonian where the model parameters can be easily
tuned by varying the lattice potential opened up a new regime in the
many-body physics of atomic gases. Unlike Bose-Einstein condensates in
free space which are described by weakly interacting theories
\cite{dalfovo1999}, the behaviour of ultracold gases trapped in an
optical lattice is dominated by atomic interactions opening the
possibility of studying strongly correlated behaviour with
unprecendented control.

The modern field of strongly correlated ultracold gases is successful
due to its interdisciplinarity \cite{lewenstein2007,
  bloch2008}. Originally condensed matter effects are now mimicked in
controlled atomic systems finding applications in areas such as
quantum information processing. A really new challenge is to identify
novel phenomena which were unreasonable to consider in condensed
matter, but will become feasible in new systems. One such direction is
merging quantum optics and many-body physics \cite{mekhov2012,
  ritsch2013}. Quantum optics has been developping as a branch of
quantum physics independently of the progress in the many-body
community. It describes delicate effects such as quantum measurement,
state engineering, and systems that can generally be easily isolated
from their environnment due to the non-interacting nature of photons
\cite{Scully}. However, they are also the perfect candidate for
studying open systems due the advanced state of cavity technologies
\cite{carmichael, MeasurementControl}. On the other hand ultracold
gases are now used to study strongly correlated behaviour of complex
macroscopic ensembles where decoherence is not so easy to avoid or
control. Recent experimental progress in combining the two fields
offered a very promising candidate for taking many-body physics in a
direction that would not be possible for condensed matter
\cite{baumann2010, wolke2012, schmidt2014}. Furthermore, two very
recent breakthrough experiments have even managed to couple an
ultracold gas trapped in an optical lattice to an optical cavity
enabling the study of strongly correlated systems coupled to quantised
light fields where the quantum properties of the atoms become
imprinted in the scattered light \cite{klinder2015, landig2016}.

There are three prominent directions in which the field of quantum
optics of quantum gases has progressed in. First, the use of quantised
light enables direct coupling to the quantum properties of the atoms
\cite{mekhov2007prl, mekhov2007pra, mekhov2007NP, mekhov2012}. This
allows us to probe the many-body system in a nondestructive manner and
under certain conditions even perform quantum non-demolition (QND)
measurements. QND measurements were originally developed in the
context of quantum optics as a tool to measure a quantum system
without significantly disturbing it \cite{braginsky1977, unruh1978,
  brune1990, brune1992}. This has naturally been extended into the
realm of ultracold gases where such non-demolition schemes have been
applied to both fermionic \cite{eckert2008qnd, roscilde2009} and
bosonic systems \cite{hauke2013, rogers2014}. In this thesis, we
consider light scattering in free space from a bosonic ultracold gas
and show that there are many prominent features that go beyond
classical optics. Even the scattering angular distribution is
nontrivial with Bragg conditions that are significantly different from
the classical case. Furthermore, we show that the direct coupling of
quantised light to the atomic systems enables the nondestructive
probing beyond a standard mean-field description. We demonstrate this
by showing that the whole phase diagram of a disordered
one-dimensional Bose-Hubbard Hamiltonian, which consists of the
superfluid, Mott insulating, and Bose glass phases, can be mapped from
the properties of the scattered light. Additionally, we go beyond
standard QND approaches, which only consider coupling to density
observables, by also considering the direct coupling of the quantised
light to the interference between neighbouring lattice sites. We show
that not only is this possible to achieve in a nondestructive manner,
it is also achieved without the need for single-site resolution. This
is in contrast to the standard destructive time-of-flight measurements
currently used to perform these measurements \cite{miyake2011}. Within
a mean-field treatment this enables probing of the order parameter as
well as matter-field quadratures and their squeezing. This can have an
impact on atom-wave metrology and information processing in areas
where quantum optics has already made progress, e.g.,~quantum imaging
with pixellized sources of non-classical light \cite{golubev2010,
  kolobov1999}, as an optical lattice is a natural source of multimode
nonclassical matter waves.

Second, coupling a quantum gas to a cavity also enables us to study
open system many-body dynamics either via dissipation where we have no
control over the coupling to the environment or via controlled state
reduction using the measurement backaction due to
photodetections. Initially, a lot of effort was exended in an attempt
to minimise the influence of the environment in order to extend
decoherence times. However, theoretical progress in the field has
shown that instead being an obstacle, dissipation can actually be used
as a tool in engineering quantum states \cite{diehl2008}. Furthermore,
as the environment coupling is varied the system may exhibit sudden
changes in the properties of its steady state giving rise to
dissipative phase transitions \cite{carmichael1980, werner2005,
  capriotti2005, morrison2008, eisert2010, bhaseen2012, diehl2010,
  vznidarivc2011}. An alternative approach to open systems is to look
at quantum measurement where we consider a quantum state conditioned
on the outcome of a single experimental run \cite{carmichael,
  MeasurementControl}. In this approach we consider the solutions to a
stochastic Schr\"{o}dinger equation which will be a pure state, which
in contrast to dissipative systems where this is generally not the
case. The question of measurement and its effect on the quantum state
has been around since the inception of quantum theory and still
remains a largely open question \cite{zurek2002}. It wasn't long after
the first condenste was obtained that theoretical work on the effects
of measurement on BECs appeared \cite{cirac1996, castin1997,
  ruostekoski1997}. Recently, work has also begun on combining weak
measurement with the strongly correlated dynamics of ultracold gases
in optical lattices \cite{mekhov2009prl, mekhov2009pra, mekhov2012,
  douglas2012, douglas2013, ashida2015, ashida2015a}.

In this thesis we focus on the latter by considering a quantum gas in
an optical lattice coupled to a cavity \cite{mekhov2012}. This
provides us with a flexible setup where the global light scattering
can be engineered. We show that this introduces a new competing energy
scale into the system and by considering continuous measurement, as
opposed to discrete projective measurements, we demonstrate the
quantum backaction can effectively compete with the standard
short-range processes of the Bose-Hubbard model. The global nature of
the optical fields leads to new phenomena driven by long-range
correlations that arise from the measurement. The flexibility of the
optical setup lets us not only consider coupling to different
observables, but by carefully choosing the optical geometry we can
suppress or enhace specific dynamical processes, realising spatially
nonlocal quantum Zeno dynamics.

The quantum Zeno effect happens when frequent measurements slow the
evolution of a quantum system \cite{misra1977, facchi2008}. This
effect was already considered by von Neumann and it has been
successfully observed in a variety of systems \cite{itano1990,
  nagels1997, kwiat1999, balzer2000, streed2006, hosten2006,
  bernu2008}. The generalisation of this effect to measurements with
multidimensional projections leads to quantum Zeno dynamics where
unitary evolution is uninhibited within this degenerate subspace,
i.e. the Zeno subspace \cite{facchi2008, raimond2010, raimond2012,
  signoles2014}. Here, by combining quantum optical measurements with
the complex Hilbert space of a many-body quantum gas we go beyond
conventional quantum Zeno dynamics. By considering the case of
measurement near, but not in, the projective limit the system is still
confined to a Zeno subspace, but intermediate transitions are allowed
via virtual Raman-like processes. In a lattice system, like the
Bose-Hubbard model we can use global measurement to engineer these
dynamics to be highly nonlocal leading to the generation of long-range
correlations and entanglement. Furthermore, we show that this
behaviour can be approximated by a non-Hermitian Hamiltonian thus
extending the notion of quantum Zeno dynamics into the realm of
non-Hermitian quantum mechanics joining the two
paradigms. Non-Hermitian systems themselves exhibit a range of
interesting phenomena ranging from localisation \cite{hatano1996,
  refael2006} and {\fontfamily{cmr}\selectfont $\mathcal{PT}$
  symmetry} \cite{bender1998, giorgi2010, zhang2013} to spatial order
\cite{otterbach2014} and novel phase transitions \cite{lee2014prx,
  lee2014prl}.

Just like for the nondestructive measurements we also consider
measurement backaction due to coupling to the interference terms
between the lattice sites. This effectively amounts to coupling to the
phase observables of the system. As this is the conjugate variable of
density, this allows to enter a new regime of quantum control using
measurement backaction. Whilst such interference measurements have
been previously proposed for BECs in double-wells \cite{cirac1996,
  castin1997, ruostekoski1997}, the extension to a lattice system is
not straightforward. However, we will show it is possible to achieve
with our proposed setup by a careful optical arrangement. Within this
context we demonstrate a novel type of projection which occurs even
when there is significant competition with the Hamiltonian
dynamics. This projection is fundamentally different to the standard
formulation of the Copenhagen postulate projection or the quantum Zeno
effect \cite{misra1977, facchi2008} thus providing an extension of the
measurement postulate to dynamical systems subject to weak
measurement.

Finally, the cavity field that builds up from the scattered photons
can also create a quantum optical potential which will modify the
Hamiltonian in a way that depends on the state of the atoms that
scatterd the light. This can lead to new quantum phases due to new
types of long-range interactions being mediated by the global quantum
optical fields \cite{caballero2015, caballero2015njp, caballero2016,
  caballero2016a, elliott2016}. However, this aspect of quantum optics
of quantum gases is beyond the scope of this thesis.

\newpage

\section*{Publication List}

The work contained in this thesis is based on seven publications
\cite{kozlowski2015, elliott2015, atoms2015, mazzucchi2016,
  kozlowski2016zeno, mazzucchi2016njp, kozlowski2016phase}:

\begin{table}[hbtp!]
  \centering
  \begin{tabular}{r p{13cm}}
    \toprule
    \cite{kozlowski2015} & W. Kozlowski, S. F. Caballero-Benitez, and
    I. B. Mekhov.  ``Probing matter-field and atom-number correlations
    in optical lattices by global nondestructive addressing''.
    \emph{Physical Review A}, 92(1):013613, 2015. \\ \\

    \cite{elliott2015} & T. J. Elliott, W. Kozlowski,
    S. F. Caballero-Benitez, and I. B. Mekhov. ``Multipartite
    Entangled Spatial Modes of Ultracold Atoms Generated and
    Controlled by Quantum Measurement''. \emph{Physical Review
      Letters}, 114:113604, 2015. \\ \\

    \cite{atoms2015} & T. J. Elliott, G. Mazzucchi, W. Kozlowski,
    S. F. Caballero- Benitez, and I. B. Mekhov. ``Probing and
    manipulating fermionic and bosonic quantum gases with quantum
    light''. \emph{Atoms}, 3(3):392–406, 2015. \\ \\

    \cite{mazzucchi2016} & G. Mazzucchi$^*$, W. Kozlowski$^*$,
    S. F. Caballero-Benitez, T. J.  Elliott, and
    I. B. Mekhov. ``Quantum measurement-induced dynamics of many- body
    ultracold bosonic and fermionic systems in optical
    lattices''. \emph{Physical Review A}, 93:023632,
    2016. $^*$\emph{Equally contributing authors}. \\ \\

    \cite{kozlowski2016zeno} & W. Kozlowski, S. F. Caballero-Benitez,
    and I. B. Mekhov. ``Non- hermitian dynamics in the quantum zeno
    limit''. \emph{Physical Review A}, 94:012123, 2016. \\ \\ 

    \cite{mazzucchi2016njp} & G. Mazzucchi, W. Kozlowski,
    S. F. Caballero-Benitez, and I. B Mekhov. ``Collective dynamics of
    multimode bosonic systems induced by weak quan- tum
    measurement''. \emph{New Journal of Physics}, 18(7):073017, 2016. \\ \\

    \cite{kozlowski2016phase} & W. Kozlowski, S. F. Caballero-Benitez,
    and I. B. Mekhov. ``Quantum state reduction by
    matter-phase-related measurements in optical
    lattices''. \emph{arXiv preprint arXiv:1605.06000}, 2016. \\

    \bottomrule
  \end{tabular}
\end{table}
