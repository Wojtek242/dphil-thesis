%*******************************************************************************
%*********************************** Seventh Chapter *****************************
%*******************************************************************************

\chapter{Summary and Conclusions}  %Title of the Seventh Chapter

\ifpdf
    \graphicspath{{Chapter7/Figs/Raster/}{Chapter7/Figs/PDF/}{Chapter7/Figs/}}
\else
    \graphicspath{{Chapter7/Figs/Vector/}{Chapter7/Figs/}}
\fi

Quantum optics of quantum gases explores the ultimate quantum regime
of light-matter interactions where both the optical and matter fields
are fully quantised. It provides a very rich system in which new
phenomena can be observed, engineered, and controlled beyond what
would be possible in condensed matter. Combined with rapid and
promising experimental progress in this field the theoretical
proposals have the potential of directing the research in the
foreseeable future \cite{baumann2010, wolke2012, schmidt2014,
  klinder2015, landig2016}.

In this thesis we focused on the coupling between global quantised
optical fields and an ultracold bosonic quantum gas. By considering
global fields as opposed to localised light-matter interactions we
were able to introduce several nonlocal properties to the Hamiltonian
in a controllable manner which would otherwise be impossible to
implement. We showed how this can be useful in the context of
nondestructive probing by showing that it can easily distinguish
between a highly delocalised quantum state such as a superfluid and
insulating states such as the Mott insulator and the Bose glass phases
which is currently a challenge \cite{derrico2014}. Furthermore, we
have seen how the correlation length, which would be inaccessible in
localised measurements, was immediately visible in our scheme and lead
to an angular scattering pattern that was far richer than it was for
the classical case. This is best highlighted by the fact that it would
be visible even when classically no light would scatter coherently at
all.

More interestingly, the global nature of the measurement was also
capable of creating such long-range correlations itself when we
considered measurement backaction. This was most visible when we saw
how weak measurement was capable of driving global macroscopic
multimode oscillations between different spatial modes, such as odd
and even sites, across the whole lattice which could be used for
quantum information and metrology. Such dynamical states show spatial
density-density correlations with nontrivial periods and long-range
coherence, thus having supersolid properties, but as an essentially
dynamical version. Furthermore, the tunability of the optical
arrangement meant that we had extreme flexibility in choosing our
observables, effectively tailoring the long-range entanglement and
correlations in the system. We have also shown how global measurement
when combined with both atomic tunnelling and interactions leads to
highly nontrivial dynamics in which backaction can either compete or
cooperate with on-site repulsion in squeezing the atomic variables.

In the limit of strong measurement when quantum Zeno dynamics occurs
we showed that these nonlocal spatial modes created by the global
measurement lead to long-range correlated tunnelling events whilst
suppressing any other dynamics between different spatial modes of the
measurement. Such globally paired tunneling due to a fundamentally
novel phenomenon can enrich physics of long-range correlated systems
beyond relatively shortrange interactions expected from standard
dipole-dipole interactions \cite{sowinski2012, omjyoti2015}. These
nonlocal high-order processes entangle regions of the optical lattice
that are disconnected by the measurement. Using different detection
schemes, we showed how to tailor density-density correlations between
distant lattice sites. Quantum optical engineering of nonlocal
coupling to environment, combined with quantum measurement, can allow
the design of nontrivial system-bath interactions, enabling new links
to quantum simulations \cite{stannigel2013} and thermodynamics
\cite{erez2008}. Interestingly, these dynamics also provide a link to
non-Hermitian quantum mechanics as this regime of measurement can be
accurately described with a non-Hermitian Hamiltonian. Furthermore, we
show that this allows for a rather novel type of competition between
measurement and tunnelling where both processes actually cooperate to
produce a steady state in which tunnelling is suppressed by
destructive matter-wave interference.

A unique feature of our global measurement scheme meant that we could
couple directly to the phase observables of the system by coupling to
the interference between the lattice sites, which represents the
shortest meaningful distance in an optical lattice, rather than their
on-site density. This defines most processes in optical lattices.  For
example, matter-field phase changes may happen not only due to
external gradients, but also due to intriguing effects such quantum
jumps leading to phase flips at neighbouring sites and sudden
cancellation of tunneling \cite{vukics2007}, which should be
accessible by this method. Furthremore, in mean-field one can measure
the matter-field amplitude (which is also the order parameter),
quadratures and their squeezing. This can link atom optics to areas
where quantum optics has already made progress, e.g., quantum imaging
\cite{golubev2010, kolobov1999}, using an optical lattice as an array
of multimode nonclassical matter- field sources with a high degree of
entanglement for quantum information processing. We have also shown
how this scheme of coupling to phase observables can be used in the
context of quantum measurement backaction to achieve a new degree of
control. We used this result to show a generalisation of weak
measurement on dynamical systems by showing that there is now a new
class of projections available even when the measurement is not a
compatible observable of the Hamiltonian. This an interesting result
as the projections themselves are unlike those postulated by the
Copenhagen interpretation, those present in quantum Zeno dynamic, or
even those possible to engineer using dissipative methods.

In this thesis we have covered significant areas of the broad field
that is quantum optics of quantum gases, but there is much more that
has been left untouched. Here, we have only considered spinless
bosons, but the theory can also been extended to fermions
\cite{atoms2015, mazzucchi2016, mazzucchi2016af} and
molecules \cite{LP2013} and potentially even photonic circuits
\cite{mazzucchi2016njp}. Furthermore, the question of quantum
measurement and its properties has been a subject of heated debate
since the very origins of quantum theory yet it is still as mysterious
as it was at the beginning of the $20^\mathrm{th}$ century. However,
this work has hopefully demonstrated that coupling quantised light
fields to many-body systems provides a very rich playground for
exploring new quantum mechanical phenomena beyond what would otherwise
be possible in other fields.
