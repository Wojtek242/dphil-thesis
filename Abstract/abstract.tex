% ************************** Thesis Abstract *****************************
% Use `abstract' as an option in the document class to print only the titlepage and the abstract.
\begin{abstract}
  Trapping ultracold atoms in optical lattices enabled the study of
  strongly correlated phenomena in an environment that is far more
  controllable and tunable than what was possible in condensed
  matter. Here, we consider coupling these systems to quantised light
  where the quantum nature of both the optical and matter fields play
  equally important roles in order to push the boundaries of
  what is possible in ultracold atomic systems.

  We show that light can serve as a nondestructive probe of the
  quantum state of the matter. By condering a global
  measurement scheme we show that it is possible to distinguish a
  highly delocalised phase like a superfluid from insulators. We also
  demonstrate that light scattering reveals not only density
  correlations, but also matter-field interference.

  By taking into account the effect of measurement backaction we show
  that the measurement can efficiently compete with the local atomic
  dynamics of the quantum gas. This can generate long-range
  correlations and entanglement which in turn leads to macroscopic
  multimode oscillations accross the whole lattice when the
  measurement is weak and correlated tunneling, as well as selective
  suppression and enhancement of dynamical processes beyond the
  projective limit of the quantum Zeno effect in the strong
  measurement regime.

  We also consider quantum measurement backaction due to the
  measurement of matter-phase-related variables such as global phase
  coherence. We show how this unconventional approach opens up new
  opportunities to affect system evolution and demonstrate how this
  can lead to a new class of measurement projections thus extending
  the measurement postulate for the case of strong competition with
  the system’s own evolution.
\end{abstract}
